\documentclass{article}
\usepackage{graphicx} % Required for inserting images

\title{Discrete Math Notes Lecture 07}
\author{Tiffany Pham}
\date{06 February 2024}

\begin{document}

\maketitle

\section{Logical Equivalence}
$\lnot(p\leftrightarrow q)\equiv p\leftrightarrow\lnot q$

\hfill \break
$\lnot(p\leftrightarrow q)\equiv\lnot((p\rightarrow q)\land(q\rightarrow p))$
\hfill \break
$\equiv\lnot(p\rightarrow q)\lor\lnot(q\rightarrow p)$
\hfill \break
$\equiv(p\land\lnot q)\lor(q\land\lnot p)$
\hfill \break
$\equiv(p\land\lnot q)\lor(\lnot p\land q)$
\hfill \break
$\equiv p\leftrightarrow\lnot q$
\hfill \break
$(\equiv\lnot p\leftrightarrow q)$

\section{Predicate Logic}
Review
\begin{itemize}
    \item Variables: x, y, z
    \item Predicates: "is greater than 3"
\end{itemize}
\underline{P(x)}: "x is greater than 3"
\hfill \break
The statement P(x) won't have a value until x is "bound"
\hfill \break
Ex: P(4) = T

P(3) = F

\hfill \break
We need \underline{quantifiers} to \underline{bind} these variables and turn \underline{predicates} into \underline{propositions}.

\hfill \break
Quantifiers will allow us to express the meaning of the English words "all" or "some".
\begin{itemize}
    \item "All men are mortal."
    \item "Some cats do not have fur."
    \item "There exists cats that do not have fur."
\end{itemize}
The two most important quantifiers are
\begin{itemize}
    \item Universal Quantifier: $\forall$ which we read as "for all"
    \item Existential Quantifier: $\exists$ which we read as "there exists"
\end{itemize}
We can now use the quantifiers to \underline{bind} the variables in statements:

$\forall$P(x): is the statement "P(x) is true for \underline{every} x"

$\exists$P(x): is the statement "P(x) is true for \underline{some} x"

\hfill \break
Note that we also need to specify the \underline{domain} of x: what values x can take.

\hfill \break
\underline{Universal Quantifier}
$\forall P(x)$ is read as "For all x, P(x)" or "For every x, P(x)"

\hfill \break
Examples:
\begin{enumerate}
    \item If P(x) denotes $"x>0"$ and the domain the set of integers then
    \begin{center}
    $\forall x P(x)$ is \underline{F}
    \end{center}
    \item If the domain is now the set of positive integers then
    \begin{center}
        $\forall x P(x)$ is \underline{T}
    \end{center}
    \item Let P(x) be "x is even" and the domain be the integers then
    \begin{center}
        $\forall x P(x)$ is \underline{F}
    \end{center}
\end{enumerate}

\hfill \break
\underline{Existential Quantifier}
\hfill \break
$\exists x P(x)$ is read as "For some x, P(x)" or "There is an x such that P(x)" or "For at least one x, P(x)"
\begin{enumerate}
    \item If P(x) denotes $"x>0"$ and U is the integers then
    \begin{center}
        $\exists xP(x)$ is \underline{T}
    \end{center}
    \item If P(x) denotes $"x<0"$ and U is the positive integers then
    \begin{center}
        $\exists xP(x)$ is \underline{F}
    \end{center}
    \item If P(x) is "x is even" and U is the integers then
    \begin{center}
        $\exists xP(x)$ is \underline{T}
    \end{center}
\end{enumerate}
\hfill \break
It can help to think about \underline{quantification} as "looping" through the elements of the domain.
\hfill \break
To evaluate $\forall xP(x)$ loop through all x in domain:
\begin{itemize}
    \item If at every step, P(x) is true then $\forall xP(x)$ is true
    \item If at a step P(x) is false then $\forall xP(x)$ is false (and the loop terminates).
\end{itemize}

\hfill \break
To evaluate $\forall xP(x)$ loop through all x in domain:
\begin{enumerate}
    \item If at some step P(x) is true then $\exists xP(x)$ is true (and loop terminates).
    \item If the loop ends without finding an x for which P(x) is true then $\exists xP(x)$ is false.
\end{enumerate}
\hfill \break
The truth value of $\exists xP(x)$ and $\forall xP(x)$ depend on both P(x) \underline{and} the domain U.

\hfill \break
\underline{Ex.}
\begin{enumerate}
    \item If U is the positive integers and P(x) is $"x<2"$ then
    \begin{itemize}
        \item $\exists xP(x)$ is \underline{T}
        \item $\forall xP(x)$ is \underline{F}
    \end{itemize}
    \item If U is now the negative integers
    \begin{itemize}
        \item $\exists xP(x)$ is \underline{T}
        \item $\forall xP(x)$ is \underline{T}
    \end{itemize}
    \item If U consists of \{3, 4, 5\} and P(x) is $"x>2"$
    \begin{itemize}
        \item $\exists xP(x)=T$
        \item $\forall xP(x)=T$
    \end{itemize}
    If P(x) is $"x<2"$
    \begin{itemize}
        \item $\exists P(x)$
        \item $\forall P(x)$
    \end{itemize}
\end{enumerate}

\section{Translating from English to predicate logic}
Often multiple ways depending on choice of domain

\hfill \break
\underline{Ex 1}: Translate to predicate logic

"Every student in this class has taken a course in Java."
\hfill \break
\underline{Solution 1:} First decide on domain U.

If U is restricted to all students in this class, define J(x) as "x has taken a course in Java."
\hfill \break
Can translate as: $\forall xJ(x)$
\hfill \break
\underline{Solution 2:} If U is all people then also define S(x) as "x is a student in this class."
\hfill \break
Can translate as $\forall x(S(x)\rightarrow J(x))$
\hfill \break
Why is the following \underline{not correct}.
\begin{center}
    $\forall x(S(x)\land J(x))$
\end{center}
\hfill \break
\underline{Ex 2} Translate to predicate logic:

"Some student in this class has taken a course in Java."

Solution 1: If U is (all) students in this class translate as
\begin{center}
    $\exists xP(x)$
\end{center}

Solution 2: If U is all people translate as
\begin{center}
    $\exists x(S(x)\land J(x))$
\end{center}

\underline{$\exists x(S(x)\rightarrow J(x))$} is \underline{not correct}.

\section{Thinking about Quantifiers as Conjunctions and Disjunctions}
If the domain is finite
\begin{itemize}
    \item Universal quantification is equivalent to conjunction of propositions.
    \item Existential quantification is equivalent to a disjunction of propositions.
\end{itemize}
\underline{Ex} If U consists of integers 1, 2, 3:

$\forall xP(x)\equiv P(1)\land P(2)\land P(3)$

$\exists xP(x)\equiv P(1)\lor P(2)\lor P(3)$

\hfill \break
\underline{Negating Quantified Expressions}
\begin{center}
    $\forall xJ(x)$
\end{center}
\hfill \break
\underline{Ex} Let J(x) be "x has taken a course in Java" and U be students in this class
\hfill \break
$\forall xJ(x)$ is "Every student in this class has taken a course in Java."
\hfill \break
Negate the statement: "It is not the case that every student in this class has taken a course in Java. Or, "There is a student in this class who has not taken Java". $\exists x\lnot J(x)$

\hfill \break
Also have
\hfill \break
$\lnot\exists xJ(x)\equiv\forall x\lnot J(x)$

\end{document}
