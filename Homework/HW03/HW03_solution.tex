\documentclass[11pt]{article}
% To produce a letter size output. Otherwise will be A4 size.
\usepackage[letterpaper]{geometry}
% For enumerated lists using letters: a. b. etc.
\usepackage{enumitem}
\usepackage{amssymb}
\topmargin -.5in
\textheight 9in
\oddsidemargin -.25in
\evensidemargin -.25in
\textwidth 7in
\begin{document}
\title{CSE 015: Discrete Mathematics\\
Spring 2024\\
Homework \#3\\
Solution}
\date{}
\maketitle
\vspace{-2cm}
% ========== Begin questions here
\begin{enumerate}
\item
\textbf{Question 1:}
\begin{enumerate}[label=(\alph*)]
\item Yes. These are the same set since it only matters whether an element is in or
not in a set. Repeating an element does not change the set.
\item Yes. Sets are unordered collections of elements. Changing the order of the
elements does not change the set.
\item No. The first set contains the element $\{1\}$ (that is, the set containing
the element 1). The second set contains the elements $1$ and $\{1\}$.
\item No. The first is the empty set. The second is the set containing the empty
set.
\end{enumerate}
\item
\textbf{Question 2:}
\begin{enumerate}[label=(\alph*)]
\item \textbf{1}. This set contains only the element $a$.
\item \textbf{1}. This set also contains only one element, the element $\{ a \}$.
\item \textbf{2}. This set contains the elements $a$ and $\{a\}$.
\item \textbf{3}. This set contains the elements $a$, $\{a\}$, and $\{a,\{a\}\}$.
\end{enumerate}
\item
\textbf{Question 3:}
The power set of a set $A$ is the set of subsets of $A$. Every power set contains,
at a minimum, two sets, the empty set and the set for which it is the power set.
Note: a power set can only have sets as its elements.
\begin{enumerate}[label=(\alph*)]
\item $\{\emptyset, \{a\} \}$
\item $\{\emptyset, \{a\}, \{b\}, \{a,b\}\}$
\item $\{ \emptyset, \{a\}, \{\{a,b\}\}, \{a,\{a,b\}\}\}$. Note that $\{a,b\}$ is
not in the power set since it is an element of $A$, not a subset of $A$.
\end{enumerate}
\item
\textbf{Question 4:}
Note that your answers must be sets. That is, they should start with $\{$ and end
with $\}$. Also, it doesn't matter which order the pairs appear in (but the
ordering of the elements in the pairs is important).
\begin{enumerate}[label=(\alph*)]
\item $\{(a,y),(a,z),(b,y),(b,z),(c,y),(c,z)\}$
\item $\{(y,a),(y,b),(y,c),(z,a),(z,b),(z,c)\}$
\item $\{(y,y),(y,z),(z,y),(z,z)\}$. Note that $(y,z)$ and $(z,y)$ are different.
\end{enumerate}
\item
\textbf{Question 5:}
It will have $m \cdot n \cdot p$ elements.
The set $A \times B \times C$ consists of ordered triples $(a, b, c)$ with $a \in
A$, $b \in B$, and $c \in C$. There are $m$ choices for the first component. For
each of these, there are $n$ choices for the second component, giving us $m\cdot n$
choices for the first two components. For each of these, there are $p$ choices for
the third component, giving us $m \cdot n \cdot p$ choices in all.
For example, in problem 4(a) above, $A$ has 3 elements and $B$ has 2 elements so $A
\times B$ has $3 \cdot 2 = 6$ elements.
\item
\textbf{Question 6:}
Note that your answers must be sets. That is, they should start with $\{$ and end
with $\}$. Also, it doesn't matter which order the elements appear in. (It also
doesn't matter if they are repeated but we typically don't repeat elements when
listing the elements in a set.)
\begin{enumerate}[label=(\alph*)]
\item $\{0,1,2,3,4,5,6\}$
\item $\{3\}$
\item $\{1,2,4,5\}$
\item $\{0,6\}$
\end{enumerate}
\item
\textbf{Question 7:}
Note that your answers must be sets. That is, they should start with $\{$ and end
with $\}$. Also, it doesn't matter which order the elements appear in.
$A=\{1,3,5,6,7,8,9\}$ and $B=\{2,3,6,9,10\}$.
You can either reason yourself to this answer or note that since $A = (A - B) \cup
(A \cap B)$, we can conclude that $A = \{1,5,7,8\} \cup \{3,6,9\} = \
{1,3,5,6,7,8,9\}$. Similarly
$B = (B - A) \cup (A \cap B) = \{2,10\} \cup \{3,6,9\} = \{2,3,6,9,10\}$.
\item
\textbf{Question 8:}
Note that your answers must be sets. That is, they should start with $\{$ and end
with $\}$.
\begin{enumerate}[label=(\alph*)]
\item $\{4,6\}$. These are the elements that appear in all three sets.
\item $\{4,5,6,8,10\}$. These are the elements that appear in either $A$ or $B$ and
also in $C$.
\end{enumerate}
\item
\textbf{Question 9:}
\begin{enumerate}[label=(\alph*)]
\item The set of all integers ($\mathbb{Z}$). This is because we are computing the
union of the sets $\{-1,0,1\}$, $\{-2,-1,0,1,2\}$, $\{-3,-2,-1,0,1,2,3\}$, $\ldots$
\item The set of all integers other than 0. This is because we are computing the
union of the sets $\{-1,1\}$, $\{-2,2\}$, $\{-3,3\}$, $\ldots$ but 0 is not in any
of these sets. (Note that we could also write the solution as $\mathbb{Z} - \
{0\}$.)
\end{enumerate}
\item
\textbf{Question 10:}
\begin{enumerate}[label=(\alph*)]
\item $\{-1,0,1\}$. This is because we are computing the intersection of the sets
$\{-1,0,1\}$, $\{-2,-1,0,1,2\}$, $\{-3,-2,-1,0,1,2,3\}$, $\ldots$ and these are the
only elements which are in all of these sets.
\item $\emptyset$. This is because we are computing the intersection of the sets $\
{-1,1\}$, $\{-2,2\}$, $\{-3,3\}$, $\ldots$ and there are no elements which appear
in all of these sets.
\end{enumerate}
\end{enumerate}
\end{document}
