\documentclass{article}
\usepackage{graphicx} % Required for inserting images

\title{Discrete Math Notes Lecture 05}
\author{Tiffany Pham}
\date{30 January 2024}

\begin{document}

\maketitle

\section{Schedule}
\begin{itemize}
    \item System specifications
    \begin{itemize}
        \item Satisfiability, unsatisfiability, validity
    \end{itemize}
    \item Puzzles
    \item Propositional equivalence
    \item Predicates and identifiers
\end{itemize}

\section{Unsatisfiability}
\underline{Unsatisfiability}: All rows of the truth table evaluate to false.

\hfill \break
Example: $p\land \lnot p$ is unsatisfiable
\begin{center}
\begin{tabular}{ |c|c|c| } 
 \hline
 p & $\lnot p$ & $p\land \lnot p$ \\ 
 \hline
 T & F & F \\ 
 F & T & F \\
 \hline
\end{tabular}
\end{center}

\hfill \break
\underline{Validity}: A proposition is \underline{valid} when it is always true for any assignment of truth values (all rows of true).

\hfill \break
Examples: $p\lor \lnot p$
\begin{center}
\begin{tabular}{ |c|c|c| } 
 \hline
 p & $\lnot p$ & $p\lor \lnot p$ \\ 
 \hline
 T & F & T \\ 
 F & T & T \\
 \hline
\end{tabular}
\end{center}

\hfill \break
$p\rightarrow p$
\begin{center}
\begin{tabular}{ |c|c| } 
 \hline
 p & $\rightarrow p$ \\
 \hline
 T & T \\ 
 F & T \\
 \hline
\end{tabular}
\end{center}

\section{Puzzles}
\textbf{Puzzle 1}: Knights and Knaves puzzles
\begin{itemize}
    \item Knights always tell the truth.
    \item Knaves always lie.
    \item You can't distinguish by appearance, only by the truth of their statements.
\end{itemize}

\hfill \break
We meet two people, A and B. They

A: B is a knight.

B: We are of the opposite kind.
\hfill \break
Can you determine their kind?

p: A is a knight \hspace{5mm} $\lnot p$: A is a Knave

q: B is a knight \hspace{5mm} $\lnot q$: B is a Knave
\hfill \break
Suppose A is a Knight

p = T \hspace{2.5mm} q = T

$(p\land \lnot q)\lor (\lnot p \land q) \rightarrow$ F

\hfill \break
If A is a Knave, so whatever they say must be a lie.

p = F

q = F

\hfill \break
There B is also a Knave.
\hfill \break
Not is is oke for B's statement to be false.

\hfill \break
\textbf{Puzzle 2}: You are presented with three boxes.
\begin{itemize}
    \item Only one box has treasure.
    \item Box 1 inscription: This box is empty.
    \item Box 2 "\hspace{3mm}": This box is empty.
    \item Box 3 "\hspace{3mm}": Treasure is in box 2.
    \item Only one inscription is correct
\end{itemize}
Q: Can you determine which box has the treasure?
\hfill \break
Let Pi = box i has treasure for i = 1, 2, 3
\hfill \break
Since only one box has treasure, we have just three cases 

1: $p_{1}=T_{1}p_{2}=F_{1}p_{3}=F$

2: $p_{1}=F_{1}p_{2}=T_{1}p_{3}=F$

3: $p_{1}=F_{1}p_{2}=F_{1}p_{3}=T$
\hfill \break
See if the last statement is correct under each of three cases.

\hfill \break
$(\lnot p_{1}\land p_{2}\land \lnot p_{2})\lor(p_{1}\land \lnot p_{2}\land \lnot p_{2})\lor(p_{1}\land p_{2}\land p_{2})$
\begin{center}
\begin{tabular}{ |c|c|c|c| } 
 \hline
 $p_{1}$ & $p_{2}$ & $p_{3}$ & $(\lnot p_{1}\land p_{2}\land \lnot p_{2})\lor(p_{1}\land \lnot p_{2}\land \lnot p_{2})\lor(p_{1}\land p_{2}\land p_{2})$ \\ 
 \hline
 T & F & F & T \\ 
 F & T & F & F \\
 F & F & T & F \\
 \hline
\end{tabular}
\end{center}

\section{Compound Propositions}
Proving that a compound proposition is satisfiable or unsatisfiable can be computationally expensive.
\hfill \break
Using truth table (exhaustive test).
\hfill \break A compound proposition with n variables will have $2^n$ rows in its truth table.
\begin{center}
\begin{tabular}{ |c|c|c| } 
 \hline
 p & q & ... \\
 \hline
 T & T & . \\ 
  &  &     \\
 \hline
\end{tabular}
\end{center}

To prove satisfiability $\rightarrow$ only need to find one row that is T.

To prove unsatisfiability, $\rightarrow$ need to check that every row is F.
\begin{center}
\begin{tabular}{ |c|c| } 
 \hline
 n & \# rows $(2^n)$ \\
 \hline
 2 & 4 \\ 
 10 & 1024 \\
 25 & $\sim3.3x10^7$ \\
 278 & $4.6x10^83$ \\
 1000 & more than can be checked by a computer in a trillion year \\
 \hline
\end{tabular}
\end{center}

$\sim3.3x10^7$ = \# seconds in a year $\sim3.3x10^7$

$4.6x10^83$ \# atoms in universe $\sim 10^78 to 10^82$


\hfill \break
\underline{Definition:} A \underline{compound proposition} is an expression made up of propositional variables and logical corrective.

\hfill \break
\underline{Definition:} If a compound proposition is always true no matter what assignment of truth values we take for its proposition, then it is called a \underline{tautology}.

\hfill \break
If the compound proposition is always false, then it is called a \underline{contradiction}.

\hfill \break
A compound proposition that is neither a tautology nor a contradiction is called a \underline{contingency}.

\hfill \break
\underline{Ex}. $a\lor \lnot a$ is a tautology
\begin{center}
\begin{tabular}{ |c|c|c| } 
 \hline
 a & $\lnot a$ & $a\lor \lnot a$ \\ 
 \hline
 T & F & T \\ 
 F & T & T \\
 \hline
\end{tabular}
\end{center}
\hfill \break
$a\lor \lnot a$ is a \underline{contradiction}.

\section{Logical Equivalence}
Definition: Compound propositions p and q called \underline{logically equivalent} if the statement $p\leftrightarrow q$ is a tautology. Logical equivalence is expressed as
\begin{displaymath}
    p\equiv q
\end{displaymath}
\begin{center}
\begin{tabular}{ |c|c|c| } 
 \hline
 a & q & $\leftrightarrow q$ \\ 
 \hline
 T & T & T \\ 
 T & F & F \\
 F & T & F \\
 F & F & T \\
 \hline
\end{tabular}
\end{center}
\hfill \break
Show that $\lnot(p\lor q)\equiv \lnot p\land \lnot q$
$\rightarrow show \lnot(p\lor q)\leftrightarrow \lnot p\land \lnot q$

\begin{center}
\begin{tabular}{ |c|c|c|c|c|c|c|c| } 
 \hline
 p & q & $\lnot p$ & $\lnot q$ & $p\lor q$ & $\lnot(p\lor q)$ & $\lnot p\land \lnot q$ & $\lnot(p\lor q)\leftrightarrow \lnot p\land \lnot q$ \\ 
 \hline
 T & T & T & F & T & F & F & T \\ 
 T & F & F & T & T & F & F & T \\
 F & T & T & F & T & F & F & T \\
 F & F & T & T & F & T & T & T \\
 \hline
\end{tabular}
\end{center}





\end{document}
