
Page
3
of 3
\documentclass[11pt]{article}
% To produce a letter size output. Otherwise will be A4 size.
\usepackage[letterpaper]{geometry}
% For enumerated lists using letters: a. b. etc.
\usepackage{enumitem}
\topmargin -.5in
\textheight 9in
\oddsidemargin -.25in
\evensidemargin -.25in
\textwidth 7in
\begin{document}
\title{\vspace{-1.5cm}CSE 015: Discrete Mathematics\\
Spring 2024\\
Homework \#3\\
Due 5:00 pm Friday, March 15 via CatCourses}
\date{}
\maketitle
\vspace{-1.5cm}
\noindent Use LaTeX to prepare your solution as a PDF. Upload \emph{both} your PDF
and your LaTeX file (hw3.pdf and hw3.tex, for example) to CatCourses. A number of
students have had trouble at the last moment figuring out how to download the .pdf
and .tex files from Overleaf for the homeworks. Make sure you know how to do this
well ahead of the homework deadline. After you upload your files to CatCourses,
make sure to download them to check that you have uploaded the correct versions.
% ========== Begin questions here
\begin{enumerate}
\item
\textbf{Question 1:}
Are these pairs of sets equal or not (you just have to answer Yes or No).
\begin{enumerate}[label=(\alph*)]
\item $\{1,3,3,3,5,5,5,5,5\}$, $\{1,3,5\}$
\item $\{5,1,3\}$, $\{1,3,5\}$
\item $\{\{1\}\}$, $\{1,\{1\}\}$
\item $\emptyset$, $\{\emptyset\}$
\end{enumerate}
\item
\textbf{Question 2:}
What is the cardinality of (the number of elements in) each of these sets?
\begin{enumerate}[label=(\alph*)]
\item $\{a\}$
\item $\{\{a\}\}$
\item $\{a,\{a\}\}$
\item $\{a,\{a\},\{a,\{a\}\}\}$
\end{enumerate}
For example, the cardinality of the set in part (c) is 2 because this set contains
two elements: $a$ and $\{a\}$.
\item
\textbf{Question 3:}
Find the power set of each of these sets, where $a$ and $b$ are distinct elements.
\begin{enumerate}[label=(\alph*)]
\item $\{a\}$
\item $\{a,b\}$
\item $\{a,\{a,b\}\}$
\end{enumerate}
For example, the power set of the set in part (b) is: $\{\emptyset, \{a\}, \{b\}, \
{a,b\}\}$
Note that your answers must be sets. That is, they should start with $\{$ and end
with $\}$.
\item
\textbf{Question 4:}
Let $A=\{a,b,c\}$ and $B=\{y,z\}$. Determine the following sets
\begin{enumerate}[label=(\alph*)]
\item $A \times B$
\item $B \times A$
\item $B \times B$
\end{enumerate}
Note that your answers must be sets. That is, they should start with $\{$ and end
with $\}$.
\item
\textbf{Question 5:}
How many different elements does $A \times B \times C$ have if $A$ has $m$
elements, $B$ has $n$ elements, and $C$ has $p$ elements?
\item
\textbf{Question 6:}
Let $A=\{1,2,3,4,5\}$ and $B=\{0,3,6\}$. Determine the following sets
\begin{enumerate}[label=(\alph*)]
\item $A \cup B$
\item $A \cap B$
\item $A - B$
\item $B - A$
\end{enumerate}
Note that your answers must be sets. That is, they should start with $\{$ and end
with $\}$.
\item
\textbf{Question 7:}
Determine the sets $A$ and $B$ if $A - B = \{1,5,7,8\}$, $B - A = \{2,10\}$, and $A
\cap B = \{3,6,9\}$
Note that your answers must be sets. That is, they should start with $\{$ and end
with $\}$.
\item
\textbf{Question 8:}
Let $A=\{0,2,4,6,8,10\}$, $B=\{0,1,2,3,4,5,6\}$, and $C=\{4,5,6,7,8,9,10\}$.
Determine the following sets
\begin{enumerate}[label=(\alph*)]
\item $A \cap B \cap C$
\item $(A \cup B) \cap C$
\end{enumerate}
Note that your answers must be sets. That is, they should start with $\{$ and end
with $\}$.
\item
\textbf{Question 9:}
Determine the set $\bigcup\limits_{i=1}^{\infty} A_{i}$ if, for every positive
integer $i$,
\begin{enumerate}[label=(\alph*)]
\item $A_i = \{-i, -i+1, \ldots, -1, 0, 1, \ldots, i-1, i \}$
\item $A_i = \{-i, i \}$
\end{enumerate}
Note that here you don't necessary need to provide the answer as an explicit set of
elements. You can also describe the set if that seems simpler.
\item
\textbf{Question 10:}
Determine the set $\bigcap\limits_{i=1}^{\infty} A_{i}$ if, for every positive
integer $i$,
\begin{enumerate}[label=(\alph*)]
\item $A_i = \{-i, -i+1, \ldots, -1, 0, 1, \ldots, i-1, i \}$
\item $A_i = \{-i, i \}$
\end{enumerate}
Note that here you don't necessary need to provide the answer as an explicit set of
elements. You can also describe the set if that seems simpler.
\end{enumerate}
\end{document}
