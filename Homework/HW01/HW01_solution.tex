
\documentclass[11pt]{article}
% To produce a letter size output. Otherwise will be A4 size.
\usepackage[letterpaper]{geometry}
% For enumerated lists using letters: a. b. etc.
\usepackage{enumitem}
\topmargin -.5in
\textheight 9in
\oddsidemargin -.25in
\evensidemargin -.25in
\textwidth 7in
\begin{document}
\author{Shawn Newsam}
\title{CSE 015: Discrete Mathematics\\
Spring 2024\\
Homework \#1\\
Solution}
\date{}
\maketitle
% ========== Begin questions here
\begin{enumerate}
\item
\textbf{Question 1:}
\begin{enumerate}[label=(\alph*)]
\item I did not buy a lottery ticket this week.
\item If I bought a lottery ticket this week, then I won the million dollar jackpot
on Friday.
\item I bought a lottery ticket this week, and I won the million dollar jackpot on
Friday.
\item I bought a lottery ticket this week if an only if I won the million dollar
jackpot on Friday.
\item If I did not buy a lottery ticket this week, then I did not win the million
dollar jackpot on Friday.
\end{enumerate}
\item
\textbf{Question 2:}
\begin{enumerate}[label=(\alph*)]
\item $r \land \lnot q$
\item $p \land q \land r$
\item $r \rightarrow p$
\item $p \land \lnot q \land r$
\item $(p \land q) \rightarrow r$
\item $r \leftrightarrow (q \lor p)$
\end{enumerate}
\item
\textbf{Question 3:}
To construct the truth table for a compound proposition, we work from the inside
out. In each step, we show the intermediate steps. In part (c), for example, we
first construct the truth tables for $p \land q$ and for $p \lor q$ and then
combine them to get the truth table for $(p \land q) \rightarrow (p \lor q)$.
\begin{enumerate}[label=(\alph*)]
\item
The truth table is (the second column is the intermediate result)
\begin{center}
\begin{tabular}{|c|c|c|}
\hline
$p$ & $\lnot p$ & $p \rightarrow \lnot p$\\
\hline
T & F & F \\
F & T & T \\
\hline
\end{tabular}
\end{center}
\item
The truth table is (the second column is the intermediate result)
\begin{center}
\begin{tabular}{|c|c|c|}
\hline
$p$ & $\lnot p$ & $p \leftrightarrow \lnot p$\\
\hline
T & F & F \\
F & T & F \\
\hline
\end{tabular}
\end{center}
\item
The truth table is (the third and fourth columns are the intermediate results)
\begin{center}
\begin{tabular}{|c|c|c|c|c|c|}
\hline
$p$ & $q$ & $p \land q$ & $p \lor q$ & $(p \land q) \rightarrow (p \lor q)$\\
\hline
T & T & T & T & T\\
T & F & F & T & T\\
F & T & F & T & T\\
F & F & F & F & T\\
\hline
\end{tabular}
\end{center}
\item
The truth table is (the third through fifth columns are the intermediate results)
\begin{center}
\begin{tabular}{|c|c|c|c|c|c|}
\hline
$p$ & $q$ & $\lnot p$ & $q \rightarrow \lnot p$ & $p \leftrightarrow q$ & $(q \
rightarrow \lnot p) \leftrightarrow (p \leftrightarrow q)$\\
\hline
T & T & F & F & T & F\\
T & F & F & T & F & F\\
F & T & T & T & F & F\\
F & F & T & T & T & T\\
\hline
\end{tabular}
\end{center}
\end{enumerate}
\item
\textbf{Question 4:}
\begin{enumerate}[label=(\alph*)]
\item ``But'' means ``and'' here: $r \land \lnot p$
\item ``Whenever'' means ``if'' here: $(r \land p) \rightarrow q$
\item $\lnot r \rightarrow \lnot q$
\item Again ``but'' means ``and'' here and the hypothesis is a conjunction: $(\lnot
p \land r) \rightarrow q$
\end{enumerate}
\item
\textbf{Question 5:}
\begin{enumerate}[label=(\alph*)]
\item
The two compound propositions are seen to be equivalent since the truth table below
shows they have the same value for all possible assignments of the propositional
variables. The fourth and sixth columns are the intermediate results.
\begin{center}
\begin{tabular}{|c|c|c|c|c|c|c|}
\hline
$p$ & $q$ & $r$ & $p \lor q$ & $(p \lor q) \lor r$ & $q \lor r$ & $p \lor (q \lor
r)$\\
\hline
T & T & T & T & T & T & T\\
T & T & F & T & T & T & T\\
T & F & T & T & T & T & T\\
T & F & F & T & T & F & T\\
F & T & T & T & T & T & T\\
F & T & F & T & T & T & T\\
F & F & T & F & T & T & T\\
F & F & F & F & F & F & F\\
\hline
\end{tabular}
\end{center}
\item
The two compound propositions are seen to be equivalent since the truth table below
shows they have the same value for all possible assignments of the propositional
variables.. The fourth and sixth columns are the intermediate results.
\begin{center}
\begin{tabular}{|c|c|c|c|c|c|c|}
\hline
$p$ & $q$ & $r$ & $p \land q$ & $(p \land q) \land r$ & $q \land r$ & $p \land (q \
land r)$\\
\hline
T & T & T & T & T & T & T\\
T & T & F & T & F & F & F\\
T & F & T & F & F & F & F\\
T & F & F & F & F & F & F\\
F & T & T & F & F & T & F\\
F & T & F & F & F & F & F\\
F & F & T & F & F & F & F\\
F & F & F & F & F & F & F\\
\hline
\end{tabular}
\end{center}
\end{enumerate}
\item
\textbf{Question 6:}
\begin{enumerate}[label=(\alph*)]
\item
The conditional statement is shown to be a tautology (is true for all possible
assignments of the propositional variables) though the following truth table. The
third through fifth columns are the intermediate results.
\begin{center}
\begin{tabular}{|c|c|c|c|c|c|}
\hline
$p$ & $q$ & $\lnot p$ & $p \lor q$ & $\lnot p \land (p \lor q)$ & $[\lnot p \land
(p \lor q)] \rightarrow q$\\
\hline
T & T & F & T & F & T \\
T & F & F & T & F & T \\
F & T & T & T & T & T \\
F & F & T & F & F & T \\
\hline
\end{tabular}
\end{center}
\item
The conditional statement is shown to be a tautology (is true for all possible
assignments of the propositional variables) though the following truth table. The
fourth through seventh columns are the intermediate results.
\begin{center}
\begin{tabular}{|c|c|c|c|c|c|c|c|}
\hline
$p$ & $q$ & $r$ & $p \rightarrow q$ & $q \rightarrow r$ & $(p \rightarrow q) \land
(q \rightarrow r)$ & $p \rightarrow r$ & $[(p \rightarrow q) \land (q \rightarrow
r)] \rightarrow (p \rightarrow r)$\\
\hline
T & T & T & T & T & T & T & T \\
T & T & F & T & F & F & F & T \\
T & F & T & F & T & F & T & T \\
T & F & F & F & T & F & F & T \\
F & T & T & T & T & T & T & T \\
F & T & F & T & F & F & T & T \\
F & F & T & T & T & T & T & T \\
F & F & F & T & T & T & T & T \\
\hline
\end{tabular}
\end{center}
\item
The conditional statement is shown to be a tautology (is true for all possible
assignments of the propositional variables) though the following truth table. The
third and fourth columns are the intermediate results.
\begin{center}
\begin{tabular}{|c|c|c|c|c|}
\hline
$p$ & $q$ & $p \rightarrow q$ & $p \land (p \rightarrow q)$ & $[p \land (p \
rightarrow q)] \rightarrow q$\\
\hline
T & T & T & T & T\\
T & F & F & F & T\\
F & T & T & F & T\\
F & F & T & F & T\\
\hline
\end{tabular}
\end{center}
\end{enumerate}
\end{enumerate}
\end{document}
