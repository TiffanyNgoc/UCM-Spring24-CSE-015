\documentclass{article}
\usepackage{graphicx} % Required for inserting images
\usepackage{mathabx}

\title{Discrete Math Notes 04}
\author{Tiffany Pham}
\date{25 January 2024}

\begin{document}

\maketitle

\section{Schedule}
\begin{itemize}
    \item Overview of LaTeX
    \item Logic and Proofs
    \begin{itemize}
        \item System specifications
        \begin{itemize}
            \item Satisfiability, unsatisfiability, validity
        \end{itemize}
        \item Puzzles
        \item Propositional equivalence
    \end{itemize}
\end{itemize}

\section{System Specifications}
Example: Vending Machine
\hfill \break
Specifications:
\begin{itemize}
    \item You have to put in money in order to get a drink.
    \item You can get a drink without putting money in.
\end{itemize}
Let:

p: put money in

q: get a drink

\hfill \break
$q \rightarrow p$
\hfill \break
$\lnot p\rightarrow q$
\begin{center}
\begin{tabular}{ |c|c|c|c|c| } 
 \hline
 p & q & $\lnot p$ & $\lnot p\land q$ & $q\rightarrow p$ \\ 
 \hline
 T & T & F & F & T \\ 
 T & F & F & F & T \\
 F & T & T & T & F \\
 F & F & T & F & T \\
 \hline
\end{tabular}
\end{center}

\hfill \break
A description is \underline{consistent if} the conjunction of propositions is \underline{satisfiable}.

\hfill \break
A \underline{proposition} is \underline{satisfiable} if there exists an assignment of truth value that makes the proposition (at least one can of truth table must be true.

\hfill \break
Example: Is $p\lor q$ satisfiable

yes, when $p\equiv T$ and $q\equiv T$
\begin{center}
\begin{tabular}{ |c|c|c| } 
 \hline
 p & q & $p\lor q$ \\ 
 \hline
 T & T & T \\ 
 T & F & T \\
 F & T & T \\
 F & F & F \\
 \hline
\end{tabular}
\end{center}

$\lnot p$  \hspace{5mm}  $p\lor q$ \hspace{5mm} $p\rightarrow q$

\hfill \break
\underline{$(\lnot p)\land(p\lor q)\land(p\rightarrow q)$} Satisfiable?
\begin{center}
\begin{tabular}{ |c|c|c|c|c|c| } 
 \hline
 p & q & $\lnot p$ & $p\lor q$ & $q\rightarrow p$ & $(\lnot p)\land(p \lor q)\land(p\rightarrow q$ \\ 
 \hline
 T & T & F & T & T & F \\ 
 T & F & F & T & F & F \\
 F & T & T & T & T & T \\
 F & F & T & F & T & F \\
 \hline
\end{tabular}
\end{center}















\end{document}
