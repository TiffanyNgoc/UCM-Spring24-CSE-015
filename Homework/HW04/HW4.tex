\documentclass[11pt]{article}

% To produce a letter size output. Otherwise will be A4 size.
\usepackage[letterpaper]{geometry}

% For enumerated lists using letters: a. b. etc.
\usepackage{enumitem}

\topmargin -.5in
\textheight 9in
\oddsidemargin -.25in
\evensidemargin -.25in
\textwidth 7in

\begin{document}

% Edit the following putting your first and last names and your lab section.
\author{Tyler Armstrong\\
Lab CSE-015-03L M 7:30-10:30pm}

% Edit the following replacing X with the HW number.
\title{CSE 015: Discrete Mathematics\\
Fall 2019\\
Homework \#4\\
Solution}

% Put today's date in the following.
\date{November 15, 2019}
\maketitle

% ========== Begin questions here
\begin{enumerate}

\item
\textbf{Question 1:}

\begin{enumerate}[label=(\alph*)]
\item
Domain: set of all non negative numbers Range: set of all real numbers

\item
Domain: set of all non negative numbers Range: set of all non negative numbers except 1  

\item 
Domain: the set of bit strings. Range: the set of all non negative numbers. 

\item
Domain: the set of bit strings. Range: the set of all non negative numbers.

\end{enumerate}

\item
\textbf{Question 2:}

\begin{enumerate}[label=(\alph*)]
\item
onto 

\item
not onto 

\item 
onto 

\item 
onto 

\item
not onto

\end{enumerate}

\textbf{Question 3:}

\begin{enumerate}[label=(\alph*)]
\item
phone numbers are one if students don't share a number.

\item
student IDs are one to one to accurate identify and catalog students.

\item 
the final grade in a class is one to one if every student gets a different grade letter.

\item 
the students home town is one to one if every student comes from a different town. 

\end{enumerate}

\textbf{Question 4:}

\begin{enumerate}[label=(\alph*)]
\item
is a bijection

\item
is not a bijection 

\item 
is not a bijection 

\item 
is a bijection 

\end{enumerate}

\end{enumerate}

\end{document}
