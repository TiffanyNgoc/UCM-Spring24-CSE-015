\documentclass{article}
\usepackage{graphicx} % Required for inserting images

\title{Discrete Math Notes Lecture 06}
\author{Tiffany Pham}
\date{01 February 2024}

\begin{document}

\maketitle

\section{Uh}
$p\rightarrow q\equiv \lnot p\lor q$

\begin{center}
\begin{tabular}{ |c|c|c|c|c|c| } 
 \hline
 $p$ & $q$ & $\lnot p$ & $p\rightarrow q$ & $\lnot p \lor q$ & $(p\rightarrow q)\leftrightarrow p\lor q$ \\ 
 \hline
 T & T & F & T & T & T \\ 
 T & F & F & F & F & T \\
 F & T & T & T & T & T \\
 F & F & T & T & T & T \\
 \hline
\end{tabular}
\end{center}

\section{Summary of Logical Equivalence}
$\lnot \lnot p\equiv p$ (Double Negation)
\hfill \break
$\lnot(p\lor q)\equiv \lnot p\land \lnot q$
\hfill \break
$\lnot(p\land q)\equiv \lnot p\lor \lnot q$

\hfill \break
Identity Laws:
\hfill \break
$p\land T\equiv p$
\hfill \break
$p\lor F\equiv p$

\hfill \break
Domination Laws:
\hfill \break
$p\lor T\equiv T$
\hfill \break
$p\land F \equiv F$

\hfill \break
Idempotent Laws
\hfill \break
$p\lor p\equiv p$
\hfill \break
$p\;and p\equiv p$

\hfill \break
Commutative Laws:
\hfill \break
$p\lor q\equiv q\lor p$
\hfill \break
$p\land q\equiv q\land p$

\hfill \break
Associative Laws:
\hfill \break
$(p\lor q)\lor r\equiv p\lor(q\lor r)$
\hfill \break
$(p\land q)\land r\equiv p\land(q\land r)$

\hfill \break
Distributive Laws:
\hfill \break
$p\lor(q\land r)\equiv(p\lor q)\land(p\lor r)$
\hfill \break
$p\land(q\lor r)\equiv(p\land q)\lor(p\land r)$

\hfill \break
Absorption Laws:
\hfill \break
$p\lor(p\land q)\equiv p$
\hfill \break
$p\land (p\lor q)\equiv p$

\section{Logical Equivalences Involving Conditional Statements}
$p\rightarrow q\equiv \lnot p\lor q$ \hspace{2mm} Conditional-Disjunction or Implication Removal

\hfill \break
$p\rightarrow q\equiv \lnot q\rightarrow \lnot p$ \hspace{2mm} Contra-positive

\hfill \break
$p\lor q\equiv \lnot p\rightarrow q$

\hfill \break
$p\land q\equiv \lnot(p\rightarrow \lnot q)$

\hfill \break
$(p\rightarrow q\land(p\rightarrow r)\equiv(p\rightarrow(q\land r)$

\hfill \break
$(p\rightarrow r)\land(q\rightarrow r)\equiv(p\lor q)\rightarrow r$

\hfill \break
$(p\rightarrow q)\lor(p\rightarrow r)\equiv p\rightarrow(q\lor r)$

\hfill \break
$(p\rightarrow r)\lor(q\rightarrow r)\equiv(p\land q)\rightarrow r$

\section{Logical Equivalences Involving Bi-conditional Statements}
$p\leftrightarrow q\equiv(p\rightarrow q)\land(q\rightarrow p)$

\hfill \break
$p\leftrightarrow q\equiv \lnot p\leftrightarrow \lnot q$

\hfill \break
$p\leftrightarrow q\equiv(p\land q)\lor(\lnot p\land \lnot q)$

\hfill \break
$\lnot(p\leftrightarrow q)\equiv p\leftrightarrow \lnot q\equiv \lnot p\leftrightarrow q$

\section{Constructing (Proving) New Logical Equivalences}
Show that $\lnot(p\lor(\lnot p\land q)\equiv \lnot p\land \lnot q$

Start with left and apply known logical equivalence to get right:

\hfill \break
$\lnot(p\lor(\lnot p\land q))\equiv \lnot p\land \lnot(\lnot p\land q)$ \hspace{5mm} (De Morgans)
\hfill \break
$\equiv \lnot p\land(\lnot \lnot p\lor \lnot q)$ \hspace{5mm} ("\hspace{2mm}")
\hfill \break
$\equiv \lnot p\land(p\lor \lnot q)$ \hspace{5mm} (Double Negation)
\hfill \break
$\equiv(\lnot p\land p)\lor(\lnot p\land \lnot q)$ \hspace{5mm} (Distributive)
\hfill \break
$\equiv F \lor(\lnot p\land \lnot q)$ \hspace{5mm} (Contradiction)
\hfill \break
$\equiv \lnot p\land \lnot q$ \hspace{5mm} (Identity)

\hfill \break
Show that $(p\land q)\rightarrow(p\lor q)\equiv T$

\hfill \break
$(p\land q)\rightarrow(p\lor q)\equiv \lnot(p\land q)\lor (p\lor q)$ (Implication Removal)

\hfill \break
$\equiv(\lnot p\lor \lnot q)\lor(p\lor q)$ (De Morgan's)
\hfill \break
$\equiv \lnot p\lor \lnot q \lor p \lor q$ (Associtative)
\hfill \break
$\equiv(\lnot p\lor p)\lor(\lnot q\lor q)$ (Communative \& Associative)
\hfill \break
$\equiv T \lor T$
\hfill \break
$\equiv T$

\hfill \break
$\lnot(p\leftrightarrow q)\equiv p\leftrightarrow \lnot q$
\hfill \break
$\lnot(p\leftrightarrow q)\equiv \lnot((p\rightarrow q)\land(q\rightarrow p))$
\hfill \break
$\equiv \lnot(p\rightarrow q)\lor \lnot(q\rightarrow p)$
\hfill \break
$\equiv(p\land \lnot q)\lor(q\land \lnot p)$

\section{Predicates and Quantifiers}
\underline{Propositional logic} cannot express the meanings of all statements in mathematics and natural language.

\hfill \break
For example, suppose we know

"All people are mortal"

and

"Socrates is a person"

then can't use propositional logic to conclude

"Socrates is mortal"

\hfill \break
Need a language that talks about \underline{objects}, their \underline{properties}, and their \underline{relatives}.

\hfill \break
\underline{Predicate Logic}

\hfill \break
Predicate Logic uses:
\begin{enumerate}
    \item Variables: x, y, z
    \item Predicates
    \item Quantifiers "for all," "there exists"
\end{enumerate}

\hfill \break
Consider the statement

$"x>3"$
\hfill \break
"x is the variable"
\hfill \break
"is greater than 3" is the predicate

\hfill \break
\underline{Propositional Functions}
\hfill \break
We can denote the statement "x is greater than 3" by p(x), where x is the variable and "is greater than 3" is the predicate.

\hfill \break
p here is called a \underline{propositional function}.
\hfill \break
Once a value has been assigned to x, the statement p(x) becomes a \underline{proposition} and truth value.

\hfill \break
\underline{Ex:}
\hfill \break
Let p(x) be $"x>3$

p(4) = T

p(3) = F
\hfill \break
Statements can have more than one variable
\hfill \break
Ex. Let Q(x, y) be the statement $"x=y+3"$

Q(1, 2) = F

Q(3, 0) = T

\hfill \break
Can combine statements using connectives from propositional logic:
\hfill \break
Propositional logic:

\hfill \break
Ex: Let p(x) denote $"x>0"$

$p(3)\lor p(-1)$ \hspace{5mm} T

$p(3)\land p(-1)$ \hspace{5mm} F

$p(3)\rightarrow p(-1)$ \hspace{5mm} F

$p(-1)\rightarrow p(3)$ \hspace{5mm} T

\hfill \break
Expressions with variables are not propositions and therefore don't have truth values.

\hfill \break
Ex:
\hfill \break
$p(3)\land p(y)$
\hfill \break
$p(x)\rightarrow p(y)$

























\end{document}
