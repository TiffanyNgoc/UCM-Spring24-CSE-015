\documentclass{article}
\usepackage{graphicx} % Required for inserting images
\usepackage{mathabx}

\title{Discrete Math Notes Lecture 03}
\author{Tiffany Pham}
\date{23 January 2024}

\begin{document}

\maketitle

\section{Schedule}
\begin{itemize}
    \item Conditional statements of propositions
    \begin{itemize}
        \item Conditional (revisited)
        \item Contra positive
        \item Converse
        \item Inverse
        \item Bioconditional
    \end{itemize}
    \item Translating English sentences into logic
    \item System specifications
\end{itemize}

\section{Conditional Statements (revisited)}
$p \rightarrow q$ is a proposition
\begin{itemize}
    \item if p then q
    \item p implies q
    \item q follows from p
    \item p only if q
    \item p is sufficient for q
    \item q is necessary for p
    \item q unless $\neg p$
\end{itemize}
\begin{center}
\begin{tabular}{ |c|c|c| } 
 \hline
 p & q & $p\rightarrow q$ \\ 
 \hline
 T & T & T \\ 
 T & F & F \\
 F & T & F \\
 F & F & F \\
 \hline
\end{tabular}
\end{center}
Example:
\hfill \break
p = "It is raining"
\hfill \break
q = "I will get wet"
\hfill \break
$p \rightarrow q$ "If it is raining, I will get wet."
\begin{itemize}
    \item p only if q
\end{itemize}
"It can have rained \underline{only if} I get wet."
\begin{itemize}
    \item p \underline{is sufficient} for q
\end{itemize}
"It is sufficient for it to have rained for me to have gotten wet."
\begin{itemize}
    \item q is necessary for p
\end{itemize}
"It is necessary for me to have gotten wet for it to have rained."

\hfill \break
Example:
\hfill \break
p = "$x=2$"
\hfill \break
q = "$x^2=4$
\hfill \break
$p\rightarrow q$: if $x=2$ then $x^2=4$
\hfill \break
p \underline{only} if q
\hfill \break
$x=2$ only if $x^2=4$

(but it is not required that $x=2$ because we can have $x=-2$)
\hfill \break
"p \underline{is sufficient} for q"
\hfill \break
$x=2$ \underline{is sufficient} for $x^2=4$
\begin{itemize}
    \item q is \underline{necessary} for p
\end{itemize}
$x^2=4$ is necessary for $x=2$
\hfill \break
Contra positive of $p\rightarrow q$ is $\neg q\rightarrow \neg p$

$p\rightarrow q\equiv \neg q\rightarrow \neg p$
\hfill \break
Proving equivalence of two propositions:
\hfill \break
Write out the truth table for both propositions and see if every row matches:
\begin{center}
\begin{tabular}{ |c|c|c|c|c|c| } 
 \hline
 p & q & $p\rightarrow q$ & $\neg q$ & $\neg p$ & $\neg q \rightarrow \neg p$ \\ 
 \hline
 T & T & T & F & F & T \\ 
 T & F & F & T & F & F \\
 F & T & T & F & T & T \\
 F & F & T & T & T & T \\
 \hline
\end{tabular}
\end{center}
\underline{Inverse} of $p\rightarrow q$ is $\neg p \rightarrow \neg q$
\hfill \break
\underline{Converse} of $p\rightarrow q$ is $q\rightarrow p$

\hfill \break
Examples:
\hfill \break
p = "Home team wins
\hfill \break
q = "It's raining"
\hfill \break
"The home team wins whenever it's raining
\hfill \break
$q\rightarrow p$
\hfill \break
Contra positive "$q\rightarrow p$"
\hfill \break
$q\rightarrow p\equiv \neg p \rightarrow \neg q$

"If the home team lost then it was not raining."

\hfill \break
Converse $p\rightarrow q$

"if the home team wins, then it is raining."

\hfill \break
Inverse $\neg q \rightarrow \neg p$

"If it is not raining then the home team doesn't win."
\hfill \break
When a \underline{conditional} and its \underline{converse} are both true then $p\leftrightarrow q$ is true

\hfill \break
\textbf{Definition}: let p and q be propositions. The bi-conditional statement $p\leftrightarrow q$ is true when p and q have the same truth values and it is false otherwise.

\begin{center}
\begin{tabular}{ |c|c|c| } 
 \hline
 p & q & $p\leftrightarrow q$ \\ 
 \hline
 T & T & T \\
 T & F & F \\
 F & T & F \\
 F & F & T \\
 \hline
\end{tabular}
\end{center}
"$p\leftrightarrow q$ p is sufficient and necessary for q"
\hfill \break
Ex: "You can fly if and only if you bought a ticket."

p = you are allowed to fly

q = you bought a ticket

$(q\rightarrow p)+(p\rightarrow q) = p\leftrightarrow q$
\hfill \break
Compare with
$q\rightarrow p$
\hfill \break
If you bought a ticket then you can fly.
\hfill \break
This allows you to fly without buying a ticket.

\hfill \break
So $p\leftrightarrow q$ is the same as $(p\rightarrow q)$\textasciicircum$(q\rightarrow p)$
\hfill \break
Let's prove they are equivalent with a truth table:
\begin{center}
\begin{tabular}{ |c|c|c|c|c|c| } 
 \hline
 p & q & $p\leftrightarrow q$ & $p\rightarrow q$ & $q\rightarrow p$ & $(p\rightarrow q)$\textasciicircum$(q\rightarrow p)$ \\ 
 \hline
 T & T & T & T & T & T \\ 
 T & F & F & F & T & F \\
 F & T & F & T & F & F \\
 F & F & T & T & T & T \\
 \hline
\end{tabular}
\end{center}

\vspace{50mm}
\section{Translating English to Logic:}
\hfill \break
Example: Translate
\hfill \break
"You can access the internet from campus if you are a CSE major or you are not a freshman."

\hfill \break
Let

p = "you can access the internet from campus"

q = "you are a CSE major"

r = "you are a freshman"

(q $\widecheck$ $\neg r$)

(q $\widecheck$ $\neg r$) $\rightarrow$ p

\hfill \break
You can not ride the roller-coaster if you are under 4ft, unless you're older than 16.

Let

q = "you can ride the roller-coaster"

r = "you are under 4ft"

s = "you are older than 16"

\hfill \break
(r $\widecheck$ $\neg s$) $\rightarrow \neg q$

\section{System Specifications}
We can use propositions to specify a set of specifications for a system. We can determine whether the system is:


\underline{Consistent:} All propositions are true under some assignment of truth valves.

\underline{Inconsistent:} It is not possible to satisfy all specifications.
\hfill \break
Ex:

p: The message is stored in the buffer.

q: The message is re-transmitted.
\hfill \break
Specification: p $\widecheck$ q, $p\neg q$, $p\rightarrow q$

\hfill \break
p $\widecheck$ q:

The message is stored in the buffer or it is re-transmitted.
\hfill \break
$\neg p$: 

The message is not stored in the buffer.
\hfill \break
$p\rightarrow q$:

If the message is stored in the buffer, then it is re-transmitted.

\hfill \break
Is it possible for such a system to exist (be consistent)? Check if some assignment of truth values (to p and q) makes all the propositions true.
\begin{center}
\begin{tabular}{ |c|c|c|c|c| } 
 \hline
 p & q & $\neg q$ & p $\widecheck$ q & $p\rightarrow q$ \\ 
 \hline
 T & T & F & T & T \\ 
 T & F & F & T & F \\
 F & T & T & T & T \\
 F & F & T & F & T \\
 \hline
\end{tabular}
\end{center}




\end{document}
