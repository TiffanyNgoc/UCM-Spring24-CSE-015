\documentclass{article}
\usepackage{graphicx} % Required for inserting images

\title{Discrete Math Notes Lecture 02}
\author{Tiffany Pham}
\date{18 January 2024}

\begin{document}

\maketitle

\section{Logic Puzzles}
Puzzle 1:
\hfill \break
Knights tell the truth always.
\hfill \break
Knives always lie.
\hfill \break
You meet 2 people, \textbf{A} and \textbf{B}
\hfill \break
\textbf{A says:} B is a knight.
\hfill \break
\textbf{B says:} we are of the opposite kind.
\hfill \break
\textbf{Question:} Can you determine the type of \textbf{A} and \textbf{B}?
\hfill \break
\hfill \break
Puzzle 2:
\hfill \break
You are presented with 3 boxes, only one box has the treasure.
\hfill \break
\textbf{Box 1 inscription:} \textit{This box is empty.}
\hfill \break
\textbf{Box 2 inscription:} \textit{This box is empty.}
\hfill \break
\textbf{Box 3 inscription:} \textit{The treasure is in box 2.}
\hfill \break
Only one inscription is true.
\hfill \break
\textbf{Question:} can you determine which box has the treasure?

\section{Logic}
Mathematical logic studies:
\begin{itemize}
    \item logical foundation, structure and laws of mathematical reasoning and proofs
    \item logic reasoning in general
\end{itemize}
Logic results and methods are applicable to many fields:
\begin{itemize}
    \item AI - automated theorm proving
    \item linguistics: Natural language processing
\end{itemize}
Definition: \underline{A proposition} is a declarative sentence that can be assigned a truth value (true or false).

Examples:
\begin{itemize}
    \item The sun is hot.
    \item The Earth is made of cheese.
    \item 2 + 2 = 3
\end{itemize}
\hfill \break
The following are not propositions:
\begin{itemize}
    \item 2 + x = 5 (we cannot determine its truth value)
    \item Please don't go away!
\end{itemize}
Any sentence which is a question or a command is not a proposition.
\hfill \break
We use letters to denote propositions
\hfill \break
Ex: p, q ,r ,s
\hfill \break
Truth values of propositions are denoted by T or F.
\hfill \break
Example:
\hfill \break
p = T
\hfill \break
q = F
\section{Logical Operators}
\textbf{Definition:} Let p be a proposition.
\hfill \break
The \underline{negation} of p, denoted \underline{$\neg p$}, is the statement:
\hfill \break
"It is not the case that p (is true)"
\hfill \break
P = "My computer runs Windows"
$\neg p$ = "It is not the case that my computer runs Windows"
\hfill \break
Truth tables for $\neg p$:
\begin{center}
\begin{tabular}{ |c|c| } 
 \hline
 p & $\neg p$  \\ 
 \hline
 T & F  \\ 
 F & T  \\ 
 \hline
\end{tabular}
\end{center}
\textbf{Definition:} Let p and q be proposition variables. The \underline{conjunction} of p and q, denoted by p\textasciicircum q, is the statement p and q.
\hfill \break
Example:
\hfill \break
p = "it is cloudy"
\hfill \break
q = "it is raining"
\hfill \break
p\textasciicircum q = "It is cloudy and it is raining"
\hfill \break
The truth table for p\textasciicircum q:
\begin{center}
\begin{tabular}{ |c|c|c| } 
 \hline
 p & q & p\textasciicircum q \\ 
 \hline
 T & T & T \\ 
 T & F & F \\
 F & T & F \\
 F & F & F \\
 \hline
\end{tabular}
\end{center}
p\textasciicircum q is an example of a \underline{compound proposition}. Compound proposition can contain any number of individual propositions.
\hfill \break
Aside: in truth tables, if we have n propositional variables, we should have 2 rows?

\hfill \break
\textbf{Definition:} let p and q be propositions. The disjunction of p and q, denoted by p\textasciicircum q, is the statement p or q. The truth value of p\textasciicircum q is true if at least one of p or q is true. (note this includes the case when both are true)
\hfill \break
Truth table for p\textasciicircum q
\begin{center}
\begin{tabular}{ |c|c|c| } 
 \hline
 p & q & p\textasciicircum q \\ 
 \hline
 T & T & T \\ 
 T & F & F \\
 F & T & F \\
 F & F & F \\
 \hline
\end{tabular}
\end{center}
Example: 
\hfill \break
p = "it is cloudy"
\hfill \break
q = "it is raining"
\hfill \break
p\textasciicircum q is true when it is cloud or raining or both.
\hfill \break
In natural language, we often use the word "or" as "exclusive if" (XOR) denoted by p $\oplus$ q

\hfill \break
\textbf{Definition:} Let p and q be propositions. The conditional statement \underline{p $\rightarrow$ q} is proposition
\begin{itemize}
    \item If p then q
    \item p implies q
    \item q follows from p
    \item p only if q
    \item p is sufficient for q
    \item q is necessary for p
    \item q unless $\neg p$
\end{itemize}
Truth table p $\rightarrow$ q
\begin{center}
\begin{tabular}{ |c|c|c| } 
 \hline
 p & q & $p \rightarrow q$ \\ 
 \hline
 T & T & T \\ 
 T & F & F \\
 F & T & F \\
 F & F & F \\
 \hline
\end{tabular}
\end{center}
Example:
\hfill \break
p = "you get an A for the final"
\hfill \break
q = "I give you \$10,000"
\hfill \break
p $\rightarrow$ q: "if you get an A for the final, then I will give you \$10,000."
\hfill \break
The conditional statement p $\rightarrow$ q can be thought of as a promise, so the overall value p $\rightarrow$ q should be false only when the promise (contract) is broken. This only happens when p is true and q is false because you kept your end of the deal and I did not fulfill my promise. If p is false (you failed to fulfill your end of the promise), we never specified what happens in that case. So whatever I decide to do (give \$10,000 or not give \$10,000). I am not breaking my promise
\hfill \break
p $\rightarrow$ q
\hfill \break
Contra positive
\hfill \break
p $\rightarrow$ q = $\neg q \rightarrow \neg p$


\end{document}
